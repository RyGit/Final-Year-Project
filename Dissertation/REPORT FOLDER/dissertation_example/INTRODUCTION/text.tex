\chapter{Introduction}\label{introduction}
\section{Background}
The aim of this project was to create a motion detection device using the Genuino 101 board that could be attached to a skateboard to record when skate tricks are performed. 

The initial motivation of this project was the interest I had in doing and embedded systems project as the general theme for my final year project at university. The topic of embedded systems is something I find particularly interesting. With the amount of smart home technologies cropping up these days and the ease in which to get user friendly hardware someone like myself can use, I wanted to create one of my own.  

With this in mind I looked into how I could create something I could use alongside my hobbies. At the time when I was thinking about a ‘Smart’ device I could make I had just started taking up skateboarding. From here I wondered if you could use a device to record the distance you have travelled on skateboard, however when it comes to being ‘good’ at skating it is not down to the distance you have travelled but down to tricks you can perform. I then started to think about a way of being able to digitally record when a trick has been performed. Which then led me to conclusion that by using an accelerometer you should be able to distinguish each trick due different changes on all 3 axes’ (X, Y and Z).

Once I figured out it might be possible I to do this I did some research on the technology available and found it was possible to buy Micro processing boards with accelerometers in built, but as well as this, Bluetooth capability. This then led me onto the idea of creating an app for a smartphone which would pair with the device and provide an interactive way for myself or another user to have this data recorded by the device presented to user in a very friendly way. 

After researching the different alternatives, I tipped the Genuino 101 Board to be the best Microprocessor as a solution to this idea. This was down to code libraries looking relatively familiar as it is derived from C and the fact there is already an available application from the company for smartphones to pair with the device and push simple commands to the board which endorses the type of technology I wanted to use for this project. 

The device sends the data it records to a paired smartphone via Bluetooth. The smartphone receiving this data has an application installed to interpret the data and determine which skate trick has been performed, as well as how many times it has been performed or attempted. I also developed the application as part of this project.

The motion is monitored and captured using an accelerometer and gyroscope. The application records each trick individually as each trick you can perform on a skateboard causes the board to move in a different way. 

With the Genuino 101 attached to a skateboard, this unique movement provides unique x, y and z-axis values which the accelerometer is able to detect and send to the smartphone via Bluetooth as the Genuino 101 has accelerometer, gyroscope and Bluetooth components already on board. I used the Arduino SDK and coding library to create an on board operating system that will start taking readings of the raw data values produced by the accelerometer on the Genuino 101 when triggered by request from the user using the smartphone application. This on board operating system is also able to handle the pairing of a smartphone device via Bluetooth and establish a protocol to send the accelerometer data by.

The application prompts the user to start a “Skate Session”, once the user has done this the application will start processing any data that it receives from the board attached to the Skateboard. During this “Skate Session,” the application receives data from the onboard motion detector; recording what tricks have been performed, how many times they have been performed and how many times they were successfully landed. All this information is then presented in real-time and then summarised for the user at the end of the “Skate Session” in a user friendly manner.

This report will address the issues presented to me in order to complete the project in the form of a literature review. As a result of this review I will explain the issues faced to me over the project’s life -cycle and evaluate the techniques I used to overcome them in order to complete it.

One of the main issues and also a key theme in this project, that will be discussed, is the creation of embedded systems and the issues faced when creating them such as Concurrency, Interrupt Handling and performance issues. Furthermore, to the typical embedded systems I will talk about the issues faced when sending data via wireless communication and investigate the problems presented by the practice of using accelerometers for motion detection.

With the issues stated above to be seen as one of half of the issues that have to be investigated for the project, I also had to investigate UI and UX as I wanted the application to be very user friendly so that it is simple for anyone to use. For this I had to do some research on the best practices to use when creating user inferences and how to create a good user experience. These practices will be discussed as part of the literature review.  

The Report will also explain in detail all the methods and processes that I went through in order to complete this project. Included in the Synthesis section is a through discussion of all practical work done for the project, this will cover aspects such as the Design work done for the application’s user interface, the coding done for the application and also the coding done to create the embedded system that will be loaded onto the Genuino 101. 

The report will then go on to evaluate this work. This evaluation will include a critique of the product’s fitness for purpose against the specification and requirements I laid out before undertaking the practical work. 

As part of this evaluation I will also asses the methods I decided to use in order to complete this project and decided whether the methods I used where the best possible solutions to the problems I faced. I will talk about the skills I already had which came as a great benefit to project as well as discuss what skills I have developed or learnt entirely from carrying out this project.
